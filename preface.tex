\chapter*{Preface}
\markboth{\MakeUppercase{Preface}}{}
\addcontentsline{toc}{chapter}{Preface}

本書は、競技プログラミングの入門のために書かれました。
プログラミングの基本をすでに知っていることが前提に書かれていますが、競技プログラミング地震への予備知識は必要ありません。

想定読者としては特にアルゴリズムを学んで、国際情報学オリンピック(IOI)や国際大学対抗プログラミングコンテスト(ICPC)に参加したいと考えている学生を対象としています。
それ以外にも競技プログラミングに興味のある人なら誰でも読むことができます。
優れた競技プログラマになるには多くの時間を要しますが、さまざまなことを学べる機会でもあります。
この本を読み、実際に問題を解き、コンテストに参加することに時間をかければ、アルゴリズムに対する理解が深まることは間違いありません。
この本は継続的に開発されています。この本に対するフィードバックはいつでも ahslaaks@cs.helsinki.fi まで送ってください。
(訳註: この本は2022/04/20時点のgithub上の原稿を元に翻訳しています。誤訳や意訳もあるので著者に連絡の際は原文を確認の上、連絡をしてください。)

(以下、原文)

The purpose of this book is to give you
a thorough introduction to competitive programming.
It is assumed that you already
know the basics of programming, but no previous
background in competitive programming is needed.

The book is especially intended for
students who want to learn algorithms and
possibly participate in
the International Olympiad in Informatics (IOI) or
in the International Collegiate Programming Contest (ICPC).
Of course, the book is also suitable for
anybody else interested in competitive programming.

It takes a long time to become a good competitive
programmer, but it is also an opportunity to learn a lot.
You can be sure that you will get
a good general understanding of algorithms
if you spend time reading the book,
solving problems and taking part in contests.

The book is under continuous development.
You can always send feedback on the book to
\texttt{ahslaaks@cs.helsinki.fi}.

\begin{flushright}
Helsinki, August 2019 \\
Antti Laaksonen
\end{flushright}
