\chapter{はじめに - Introduction}

競技プログラミングとは大きく2つの段階に分かれます。
アルゴリズムの設計アルゴリズムの実装です。

\key{アルゴリズムの設計} は問題解決と数学的思考です。
問題を分析し、想像力を活用して解決することが求められます。
そのアルゴリズムは正確で効率的であることが求められ、
多くの問題でポイントでは効率的なアルゴリズムの考察が重要となります。

アルゴリズムの理論的な知識は、競技プログラミングに取り組むにあたり重要な要素です。
一般的な問題は、よく知られたテクニックと新しい考察の組み合わせが必要です。
また、競技プログラミングに登場するテクニックは、アルゴリズムの科学的な研究の基礎でもあります。

\key{アルゴリズムの実装} にはプログラミングスキルが重要です。
競技プログラミングでは、実装したアルゴリズムはテストケースでテストすることで採点されるため、
アルゴリズムの考察だけではなく、その実装も正しくなければなりません。

コンテストでの良いコーディングスタイルとはシンプルで簡潔なものです。
コンテストの時間は限られるため、プログラムは素早く書かなければなりません。
通常のソフトウェア開発とは異なり、プログラムは短く(どんなに長くても数百行程度)、
コンテスト後にメンテナンスする必要はもありません。

\section{プログラミング言語}

\index{プログラミング言語}

現在(2018年)最も使われているプログラミング言語はC++, Python, Javaです。
Google Code Jam 2017の上位3000人をみてみると、
79 \% が C++,
16 \% が Python,
8 \% が Java を使用しています\cite{goo17}.
また、複数の言語を使い分けている参加者もいます。

C++が競技プログラミングに最適と考える人は多く、C++はどのコンテストでも利用できます。
C++を使う利点は非常に効率的な言語であり
標準ライブラリにはデータ構造やアルゴリズムが豊富に揃っていることです。

一方で、複数の言語を使いこなしてそれぞれの強みを理解するのも良いアプローチです。
例えば、問題に(64bitや128bitを超える)大きな整数が必要な場合、
Pythonは標準で大きな整数を扱えるため良い選択肢になります。
ただし、コンテストの問題多くではあるプログラミング言語を選択したことで
アンフェアにならないようにされています。

本書で紹介するプログラムはC++11に準拠しています。
これは、多くのプログラミングコンテストで使うことができます
(訳註:2018年はC++17対応のコンテストサイトは少なかった)。
C++には標準ライブラリのデータ構造やアルゴリズムが多く揃っており、
あなたがC++でプログラミングをしたことがなかったとしても、今が勉強を始める良い機会です!

\subsubsection{C++のテンプレート}

C++での競技プログラミング用テンプレートを以下に示します。

\begin{lstlisting}
#include <bits/stdc++.h>

using namespace std;

int main() {
    // solution comes here
}
\end{lstlisting}

最初の\texttt{\#include}は\texttt{g++}の機能で標準ライブラリを一括で読み込むことができます。
多くのコードでよく使う\texttt{iostream},
\texttt{vector}や\texttt{algorithm},
を個別にインクルードせずに使えるようになります。

\texttt{using}行は標準ライブラリの機能を使えるようにします。
\texttt{using}がない場合は
\texttt{std::cout}と書かないといけませんが、これがあることで
\texttt{cout}だけで十分になります。

このコードは以下のようにコンパイルします。

\begin{lstlisting}
g++ -std=c++11 -O2 -Wall test.cpp -o test
\end{lstlisting}

このコマンドは\texttt{test.cpp}から
\texttt{test}という実行形式のバイナリを作成します。
(\texttt{-std=c++11})はC++11としてコンパイルすることを、
(\texttt{-O2})は最適化を行うことを、
(\texttt{-Wall})は全てのWarningを出力することを意味します。

\section{入出力}

\index{入出力}

ほとんどのコンテストでは標準入出力ストリームが用いられます。

C++では標準入出力には、
入力に\texttt{cin}が使われ、出力に\texttt{cout}が使われます。
Cの関数である
\texttt{scanf}と\texttt{printf}も利用できます。

通常、入力はスペースと改行で区切られた数字と文字列で構成されており、
これらは以下のように
\texttt{cin}で入力ストリームから読み込むことができます。

\begin{lstlisting}
int a, b;
string x;
cin >> a >> b >> x;
\end{lstlisting}

cinは各要素の間に少なくとも1つのスペースか改行があることを前提に動作します。
つまり、このコードは次の両方の入力を読み取ることができます。

\begin{lstlisting}
123 456 monkey
\end{lstlisting}
\begin{lstlisting}
123    456
monkey
\end{lstlisting}
The \texttt{cout}は次のように出力に使います。
\begin{lstlisting}
int a = 123, b = 456;
string x = "monkey";
cout << a << " " << b << " " << x << "\n";
\end{lstlisting}

入出力は時として実行時間のボトルネックになります。
以下を用いることで効率的な入出力が可能です。

\begin{lstlisting}
ios::sync_with_stdio(0);
cin.tie(0);
\end{lstlisting}

\texttt{"\textbackslash n"}用いると
\texttt{endl}よりも高速です。
なぜなら、\texttt{endl}は毎回、出力のflushを行うからです。

cinとcoutに代わり、C言語では\texttt{scanf}と\texttt{printf}が存在します。
通常、これらの関数は少し高速に動作しますが、使用するのが少し複雑になります。
次のコードは入力から2つの整数を読み取ります。
\begin{lstlisting}
int a, b;
scanf("%d %d", &a, &b);
\end{lstlisting}
また、次の通り出力することができます。
\begin{lstlisting}
int a = 123, b = 456;
printf("%d %d\n", a, b);
\end{lstlisting}

問題によっては、文字列を空白ごと読み込みたいことがあり、
これは\texttt{getline}関数が利用できます。

\begin{lstlisting}
string s;
getline(cin, s);
\end{lstlisting}

入力の文字列の数がわからない場合は次のように処理することもできます。
\begin{lstlisting}
while (cin >> x) {
    // code
}
\end{lstlisting}
こうすることで入力に利用可能なデータがなくなるまで、入力から次々と要素を読み込みます。

いくつかのシステムではファイルの入出力を行う必要があります。
この場合、以下のように書くことで標準入出力と同じように操作をすることができます。
\begin{lstlisting}
freopen("input.txt", "r", stdin);
freopen("output.txt", "w", stdout);
\end{lstlisting}
この後、入力は''input.txt''から行われ、
出力は''output.txt''に対して行われます。

\section{数字を扱う - Working with numbers}

\index{整数 - integer}

\subsubsection{整数 - Integers}

最もよく使われる整数型は\texttt{int}という32ビットの整数型です。
値の範囲は$-2^{31} \ldots 2^{31}-1$です。競技プログラミングでは
$-2 \cdot 10^9 \ldots 2 \cdot 10^9$と考えても良いでしょう。
\texttt{int} 型では不十分な場合は, 
64ビット型の\texttt{long long}を使用することができます.この型の値域は
$-2^{63} \ldots 2^{63}-1$で、
同様におおよそ、$-9 \cdot 10^{18} \ldots 9 \cdot 10^{18}$として考えられます。

次のコードでは\texttt{long long}変数を定義しています。

\texttt{long long} variable:
\begin{lstlisting}
long long x = 123456789123456789LL;
\end{lstlisting}
数字の後の\texttt{LL}はこの数字は
\texttt{long long}であるということです。

\texttt{long long}を使う時の注意は
\texttt{int} との演算です。
次の計算はうまくいきません。

\begin{lstlisting}
int a = 123456789;
long long b = a*a;
cout << b << "\n"; // -1757895751
\end{lstlisting}

変数\texttt{b}は\texttt{long long}なのですが、
\texttt{a*a}は\texttt{int}同士の掛け算で、この結果は\texttt{int}として計算されてから、
\texttt{long long}に格納されます。
このため、\texttt{int}で表現できる範囲を超えてしまうので、誤った結果が格納されます。
この例では\texttt{a} を \texttt{long long} に変更するか、
\texttt{(long long)a*a}というように結果をキャストするコードにすることで正しく動作します。

多くのコンテストの問題は、\texttt{long long}型で十分なように設定されています。
時折\texttt{g++}コンパイラが128ビットの \texttt{\_\_int128\_t}型も提供していることを知っておくと,
値域が$-2^{127} \ldots 2^{127}-1$、つまり、 $-10^{38} \ldots 10^{38}$の範囲を扱えて便利です。
ただし、この型がすべてのコンテストで利用できるわけではありません。(訳註:また多少低速になることもあります)

\subsubsection{モジュロ演算 - Modular arithmetic}

\index{剰余 - remainder}
\index{モジュロ演算 - modular arithmetic}

$x \bmod m$というのは
$x$を$m$で割った時のあまりといえます。
例えば、$17 \bmod 5 = 2$で、$17 = 3 \cdot 5 + 2$と表せるためです。

答えの数が非常に大きくなる問題では「$m$で割った数を答えよ」といった問題がよく出ます。
例えば、$10^9+7$などがあります。
これにより答えが非常に大きくても
\texttt{int} や \texttt{long long}の値域で十分に扱えます。

ここで重要な性質は足し算・引き算・掛け算において、
演算の前に余りを取ることができることです。
\[
\begin{array}{rcr}
(a+b) \bmod m & = & (a \bmod m + b \bmod m) \bmod m \\
(a-b) \bmod m & = & (a \bmod m - b \bmod m) \bmod m \\
(a \cdot b) \bmod m & = & (a \bmod m \cdot b \bmod m) \bmod m
\end{array}
\]
つまり、演算のたびに余りを取れば数字がオーバーフローするほど大きくなりすぎることがありません。

$n$までの階乗である$n!$をmod $m$で求めてみましょう。
\begin{lstlisting}
long long x = 1;
for (int i = 2; i <= n; i++) {
    x = (x*i)%m;
}
cout << x%m << "\n";
\end{lstlisting}

通常の余りで$0\ldots m-1$で表現されます。
ところが、C++やいくつかの言語ではマイナスの数に対する余りは
負の数になってしまうため工夫が必要です。

これは、あまりを求め、マイナスであった場合は$m$を加算することで解決できます。
\begin{lstlisting}
x = x%m;
if (x < 0) x += m;
\end{lstlisting}
これはあくまで負の可能性となることがある場合にのみ必要です。

\subsubsection{浮動小数点数 - Floating point numbers}

\index{浮動小数点数 - floating point number}

競技プログラミングで多く用いられるのは
64-bitの浮動小数点型\texttt{double}です。
\texttt{g++}の拡張では
80-bit 浮動小数点型の\texttt{long double}を用いることができます。
ほとんどのケースでは\texttt{double}で十分ですが、
\texttt{long double} の方がより正確な値を表現できます。

答えに必要な精度は、問題文の中で示されています。
桁を制限して答えを出力する簡単な方法は,\texttt{printf}関数を使い,
書式文字列で小数点以下の桁数を与える方法でしょう。

例えば,次のコードは,$x$の値を小数点以下9桁で表示します.
\begin{lstlisting}
printf("%.9f\n", x);
\end{lstlisting}

浮動小数点数を使う際の留意点は
浮動小数点数として正確に表現できない数値があって丸め誤差が発生することです。

例えば、次のようなコードの結果は意外に思えるでしょう。
\begin{lstlisting}
double x = 0.3*3+0.1;
printf("%.20f\n", x); // 0.99999999999999988898
\end{lstlisting}

この回答は1ですが、丸め誤差のために$x$の値が1より少し小さくなってしまいました。
この状態で\texttt{==}演算子で浮動小数点数を比較すると、
精度誤差のために本来等しいはずの値が等しくなくなる可能性があります。

浮動小数点数の比較に適した方法は、
2つの数値の差が$\varepsilon$($\varepsilon$は小さな数)より小さければ等しいと仮定することです。

実際には、次のように数値を比較することができます($\varepsilon=10^{-9}$とした場合)。
\begin{lstlisting}
if (abs(a-b) < 1e-9) {
    // a and b are equal
}
\end{lstlisting}

尚、ある程度までの整数は正確に表現でき、
\texttt{double}の場合は$2^{53}$までの整数は正しく表現できます。

\section{短く書く - Shortening code}

競技プログラミングでは、
限られた時間で素早くプログラムを書くことが求められるので、短いコードが理想的です。
そのため、競技プログラマは、
データ型などのコードに短い名前を定義することがよくあります。

\subsubsection{型の名前 - Type names}
\index{tuppdef@\texttt{typedef}}
\texttt{typedef}を用いるとデータ型を短くできます。
例えば、次のように\texttt{long long}という長い名前を
\texttt{ll}とすることができます。

\begin{lstlisting}
typedef long long ll;
\end{lstlisting}

このように宣言することで
\begin{lstlisting}
long long a = 123456789;
long long b = 987654321;
cout << a*b << "\n";
\end{lstlisting}
というコードを次のように書けます。
\begin{lstlisting}
ll a = 123456789;
ll b = 987654321;
cout << a*b << "\n";
\end{lstlisting}

\texttt{typedef}はもう少し複雑な型も表現できます。
例えば次のように、整数のvectorや、整数のpairを宣言できます。
\begin{lstlisting}
typedef vector<int> vi;
typedef pair<int,int> pi;
\end{lstlisting}

\subsubsection{マクロ -  Macros}
\index{マクロ - macro}

\key{マクロ - macros}も短く書くのに有効な手法です。
マクロは、コード中の特定の文字列をコンパイル前に変更します。
C++ では、マクロは\texttt{\#define} キーワードを使って定義します。

例えば、次のようなマクロを定義することができます。
\begin{lstlisting}
#define F first
#define S second
#define PB push_back
#define MP make_pair
\end{lstlisting}

これによって次のようなコードが、
\begin{lstlisting}
v.push_back(make_pair(y1,x1));
v.push_back(make_pair(y2,x2));
int d = v[i].first+v[i].second;
\end{lstlisting}
次のようにかけるのです。
\begin{lstlisting}
v.PB(MP(y1,x1));
v.PB(MP(y2,x2));
int d = v[i].F+v[i].S;
\end{lstlisting}

また、マクロは引数を持てるのでループなどを短く書くのに役立ちます。

\begin{lstlisting}
#define REP(i,a,b) for (int i = a; i <= b; i++)
\end{lstlisting}
このような定義をしておけば、
\begin{lstlisting}
for (int i = 1; i <= n; i++) {
    search(i);
}
\end{lstlisting}
この記述が以下のように記載できます。
\begin{lstlisting}
REP(i,1,n) {
    search(i);
}
\end{lstlisting}

ただ、マクロは時折デバッグを難しくすることがあるので注意してください。
よくある例は次の者です。
\begin{lstlisting}
#define SQ(a) a*a
\end{lstlisting}
このような平方根を求めるマクロを書いたとします。
\begin{lstlisting}
cout << SQ(3+3) << "\n";
\end{lstlisting}
次のように展開されます。
\begin{lstlisting}
cout << 3+3*3+3 << "\n"; // 15
\end{lstlisting}

このマクロは次のように書くべきでした。
\begin{lstlisting}
#define SQ(a) (a)*(a)
\end{lstlisting}
すると
\begin{lstlisting}
cout << SQ(3+3) << "\n";
\end{lstlisting}
この処理は次のように展開されるので予想した通りに動きますね。
\begin{lstlisting}
cout << (3+3)*(3+3) << "\n"; // 36
\end{lstlisting}


\section{数学 - Mathematics}

競技プログラミングで数学は重要な役割を担っており、
数学の能力がなければ競技用プログラマとして成功することは困難でしょう。
この本の後半で必要となる重要な数学的概念と公式について説明します。

\subsubsection{加算に関する公式 - Sum formulas}

まず基本的な式を紹介します。
\[\sum_{x=1}^n x^k = 1^k+2^k+3^k+\ldots+n^k,\]
ここで、$k$は正の整数であり、次数$k + 1$の多項式となる閉形式を持ちます。
他にも以下のようなものがあります。\footnote{\index{Faulhaber's formula}
There is even a general formula for such sums, called \key{Faulhaber's formula},
but it is too complex to be presented here.},
\[\sum_{x=1}^n x = 1+2+3+\ldots+n = \frac{n(n+1)}{2}\]
や
\[\sum_{x=1}^n x^2 = 1^2+2^2+3^2+\ldots+n^2 = \frac{n(n+1)(2n+1)}{6}.\]
です。

\index{等差数列 - arithmetic progression}

\key{等差数列 - arithmetic progression}は隣り合う2つの要素の差が一定であるものです。
\[3, 7, 11, 15\]
この場合、差は常に4です。等差数列の和は次の式で表されます。
\[\underbrace{a + \cdots + b}_{n \,\, \textrm{numbers}} = \frac{n(a+b)}{2}\]
$a$ は初項、
$b$ は最後の項、
$n$ は項の数です.
例えば次のようになります。
\[3+7+11+15=\frac{4 \cdot (3+15)}{2} = 36.\]
これは、和が$n$個の数からなり、
各数の値が平均して$(a+b)/2$であることから算出されます。

\index{等比数列 - geometric progression}
\key{等比数列 - geometric progression}
は隣り合う2つの値が決まった比により求められるものです。
\[3,6,12,24\]
というのは常に2倍になっています。これは次のように求められます。
\[a + ak + ak^2 + \cdots + b = \frac{bk-a}{k-1}\]
$a$ は初項、
$b$ は最後の項、
$k$ は項の比です。
\[3+6+12+24=\frac{24 \cdot 2 - 3}{2-1} = 45.\]

どのように求められるかをみていきます。
\[ S = a + ak + ak^2 + \cdots + b .\]
両辺に$k$をかけます。
\[ kS = ak + ak^2 + ak^3 + \cdots + bk,\]
これを整理して以下のように明らかです。
\[ kS-S = bk-a\]

特殊なケースでは以下の公式が成り立ちます。
\[1+2+4+8+\ldots+2^{n-1}=2^n-1.\]

\index{調和級数 - harmonic sum}

\key{調和級数 - harmonic sum} は次のような式のことです。
\[ \sum_{x=1}^n \frac{1}{x} = 1+\frac{1}{2}+\frac{1}{3}+\ldots+\frac{1}{n}.\]

調和和の上限は$\log_2(n)+1$です。
すなわち、$k$が$k$を超えないような最も近い2の累乗になるように各項$1/k$を調整すればよいです。
例えば、$n = 6$のとき、次のように見積もれます。

\[ 1+\frac{1}{2}+\frac{1}{3}+\frac{1}{4}+\frac{1}{5}+\frac{1}{6} \le
1+\frac{1}{2}+\frac{1}{2}+\frac{1}{4}+\frac{1}{4}+\frac{1}{4}.\]

この時の上限は$\log_2(n)+1$($1$, $2 \cdot 1/2$, $4 \cdot 1/4$, など)
であり、それぞれは最大で1です。

\subsubsection{集合論 - Set theory}

\index{集合論 - set theory}
\index{集合 - set}
\index{集合和 - intersection}
\index{集合積 - union}
\index{集合差 - difference}
\index{部分集合 - subset}
\index{全体集合 - universal set}
\index{補集合 - complement}

\key{集合 - set}は要素のコレクションのことです。
例えば集合である
\[X=\{2,4,7\}\]
は 2, 4, 7を要素として持ちます。
特殊なシンボル$\emptyset$は空集合を意味します。
$|S|$は集合に含まれる要素の数を示し、先ほどの例なら$|X|=3$です。

ある集合$S$に要素$x$が含まれている場合は$x \in S$
と、含まれていない場合は $x \notin S$と示します。
例をあげると次のようになります。
\[4 \in X \hspace{10px}\textrm{and}\hspace{10px} 5 \notin X.\]

\begin{samepage}
集合の演算により新しい集合を得ることができます。
\begin{itemize}
\item \key{集合積 - intersection} $A \cap B$ は$A$, $B$両方に含まれている要素の集合です。
$A=\{1,2,5\}$ , $B=\{2,4\}$のとき、 $A \cap B = \{2\}$.
\item The \key{集合和 - union} $A \cup B$ は $A$, $B$いずれかに含まれている要素の集合です。
$A=\{3,7\}$ , $B=\{2,3,8\}$ のとき、$A \cup B = \{2,3,7,8\}$.
\item The \key{補集合 - complement} $\bar A$ は $A$に含まれていない要素です。
補集合はこの集合を考える際の\key{全体集合 - universal set}に
依存します。例えば、$A=\{1,2,5,7\}$ であるときに全体の集合が
$\{1,2,\ldots,10\}$なのであれば $\bar A = \{3,4,6,8,9,10\}$となります。
\item The \key{差集合 - difference} $A \setminus B = A \cap \bar B$
は$A$ に含まれているが $B$に含まれていないものです.
ここで、$B$には$A$に含まれていない要素が含まれることもあります。
$A=\{2,3,7,8\}$ で $B=\{3,5,8\}$とするとき、
$A \setminus B = \{2,7\}$です。
\end{itemize}
\end{samepage}

$A$の各要素が$S$にも属する場合、AはSの\key{部分集合 - subset}であると呼べて、
$A \subset S$と表記します。
集合$S$は常に、空集合を含む$2^{|S|}$ 個の部分集合を持ちます。
例えば、集合 $\{2,4,7\}$の部分集合は次の通りになります。
\begin{center}
$\emptyset$,
$\{2\}$, $\{4\}$, $\{7\}$, $\{2,4\}$, $\{2,7\}$, $\{4,7\}$ and $\{2,4,7\}$.
\end{center}

よく、次のような集合が使われます。
$\mathbb{N}$ (自然数 - natural numbers),
$\mathbb{Z}$ (整数 - integers),
$\mathbb{Q}$ (有理数 - rational numbers) and
$\mathbb{R}$ (実数 - real numbers).
このうち$\mathbb{N}$は0を含むかは問題によって異なり、
$\mathbb{N}=\{0,1,2,\ldots\}$である場合もあれば、
$\mathbb{N}=\{1,2,3,...\}$となる場合もあることに注意します。

次のような集合を定義しても良いです。
\[\{f(n) : n \in S\},\]
$f(n)$はなんらかの関数です。
これは、$S$に含まれる要素$n$を入力とした関数$f(n)$の結果全ての要素を含みます。
\[X=\{2n : n \in \mathbb{Z}\}\]
は全ての偶数を含むことになります。


\subsubsection{論理式 - Logic}

\index{論理式 - logic}
\index{negation}
\index{conjuction}
\index{disjunction}
\index{implication}
\index{equivalence}

論理式の値は
\key{true} (1) か \key{false} (0)で表されます(訳注: 日本語では真と偽ともいいます)。
代表的な論理式の記号を示します。
$\lnot$ (\key{negation}),
$\land$ (\key{conjunction}),
$\lor$ (\key{disjunction}),
$\Rightarrow$ (\key{implication}),
$\Leftrightarrow$ (\key{equivalence}).

\begin{center}
\begin{tabular}{rr|rrrrrrr}
$A$ & $B$ & $\lnot A$ & $\lnot B$ & $A \land B$ & $A \lor B$ & $A \Rightarrow B$ & $A \Leftrightarrow B$ \\
\hline
0 & 0 & 1 & 1 & 0 & 0 & 1 & 1 \\
0 & 1 & 1 & 0 & 0 & 1 & 1 & 0 \\
1 & 0 & 0 & 1 & 0 & 1 & 0 & 0 \\
1 & 1 & 0 & 0 & 1 & 1 & 1 & 1 \\
\end{tabular}
\end{center}

$\lnot A$ は$A$と反対の値です。
$A \land B$ は$A$ ,  $B$が真なら真です。
$A \lor B$ は $A$ ,  $B$のどちらか(両方を含む)が真なら真です。
$A \Rightarrow B$ は、$A$が真のとき$B$が真であるとき真で、$A$が偽なら$B$はどちらでも良いです。
$A \Leftrightarrow B$ は$A$, $B$が両方同じ値のとき真です。

\index{論理述語 - predicate}

\key{論理述語 - predicate}とは、
そのパラメータによって真または偽を返す式です。
述語は通常、大文字で表記します。
例えば、$x$が素数のときだけ真になる述語$P(x)$が考えられます。
この定義によれば、$P(7)$は真だが、$P(8)$は偽を返します。

\index{量化子 - quantifier}

\key{量化子 - quantifier}
は論理式と集合の要素を結びつけるものです。
最も重要な量詞は
$\forall$ (\key{すべての - for all}) and $\exists$ (ある - \key{there is}).
です。例えば
\[\forall x (\exists y (y < x))\]
は,集合の各要素 $x$ に対して,$y$ が $x$ よりも小さいような要素 $y$ が集合に存在する、を表現できます。

このような表記法を用いると、
さまざまな論理命題を表現することができます。

\[\forall x ((x>1 \land \lnot P(x)) \Rightarrow (\exists a (\exists b (a > 1 \land b > 1 \land x = ab))))\]
というのは、ある数$x$が1より大きくて素数でないなら、
1より大きくてその積が$x$である数$a$と$b$が存在する、という命題を表現できるのです。

\subsubsection{関数 - Functions}

$\lfloor x \rfloor$は$x$を切り捨てる関数で、
$\lceil x \rceil$は$x$を切り上げる関数です。
\[ \lfloor 3/2 \rfloor = 1 \hspace{10px} \textrm{and} \hspace{10px} \lceil 3/2 \rceil = 2.\]

$\min(x_1,x_2,\ldots,x_n)$と、$\max(x_1,x_2,\ldots,x_n)$
はそれぞれ$x_1,x_2,\ldots,x_n$の最大値と最小値を返す関数です。
\[ \min(1,2,3)=1 \hspace{10px} \textrm{and} \hspace{10px} \max(1,2,3)=3.\]

\index{階乗 - factorial}

\key{階乗 - factorial} である $n!$は次のことです。
\[\prod_{x=1}^n x = 1 \cdot 2 \cdot 3 \cdot \ldots \cdot n\]
次のように再帰で表現しても良いです。
\[
\begin{array}{lcl}
0! & = & 1 \\
n! & = & n \cdot (n-1)! \\
\end{array}
\]

\index{フィボナッチ数 - Fibonacci number}

\key{フィボナッチ数 - Fibonacci numbers}
%\footnote{Fibonacci (c. 1175--1250) was an Italian mathematician.}
は色々なところで登場する関数で次のように定義されます。
\[
\begin{array}{lcl}
f(0) & = & 0 \\
f(1) & = & 1 \\
f(n) & = & f(n-1)+f(n-2) \\
\end{array}
\]
これらの数列は次のようになります。
\[0, 1, 1, 2, 3, 5, 8, 13, 21, 34, 55, \ldots\]
フィボナッチ数は閉形式が存在します。
これは\index{ビネの公式 - Binet's formula} \key{ビネの公式 - Binet's formula}
と呼ばれています。
\[f(n)=\frac{(1 + \sqrt{5})^n - (1-\sqrt{5})^n}{2^n \sqrt{5}}.\]

\subsubsection{対数 - Logarithms}

\index{対数 - logarithm}

\key{対数 - logarithm}とはある$x$を考えて
$\log_k(x)$で示されるもので、
$k$は底と呼ばれます。
定義は$\log_k(x)=a$のとき、$k^a=x$です。

対数の便利な性質は、$\log_k(x)$というのが、
$x$を$k$で1にを切らないまでの回数を表すことです。
例えば、$\log_2(32)=5$は、2で5回割ることが出来ます。
\[32 \rightarrow 16 \rightarrow 8 \rightarrow 4 \rightarrow 2 \rightarrow 1 \]

アルゴリズムの解析には対数がよく使われます。
多くの効率的なアルゴリズムは、
各ステップで何かを半分にするためです。
したがって、対数を用いてそのようなアルゴリズムの効率を推定することができます。

対数の公式を見ていきます。
\[\log_k(ab) = \log_k(a)+\log_k(b),\]
\[\log_k(x^n) = n \cdot \log_k(x).\]
\[\log_k\Big(\frac{a}{b}\Big) = \log_k(a)-\log_k(b).\]
\[\log_u(x) = \frac{\log_k(x)}{\log_k(u)},\]
最後の式は任意の底への対数を計算することが可能にします。


\index{自然対数 - natural logarithm}

\key{自然対数 - natural logarithm} $\ln(x)$ とは、
$e \approx 2.71828$を底とする対数です。

対数のもう一つの性質は、
底が$b$の整数$x$の桁数は$\lfloor \log_b(x)+1 \rfloor$であることです
(訳注: $b$進数、と考えてよいです)
。
例えば、$123$を底$2$で表すと1111011となり、
$\lfloor \log_2(123)+1 \rfloor = 7$となります。

Another property of logarithms is that
the number of digits of an integer $x$ in base $b$ is
$\lfloor \log_b(x)+1 \rfloor$.
For example, the representation of
$123$ in base $2$ is 1111011 and
$\lfloor \log_2(123)+1 \rfloor = 7$.

\section{コンテストや各種情報 - Contests and resources}

\subsubsection{IOI}

国際情報オリンピック(IOI)は、
毎年開催される中高生を対象としたプログラミングコンテストです。
各国は4人の学生からなるチームを派遣でき、例年80カ国から約300名の参加者があります。

IOIは5時間に及ぶ2つのコンテストで構成されています。
両コンテストとも参加者は様々な難易度の3つのアルゴリズム課題を解くよう求められます。
タスクはサブタスクに分けられ、それぞれにスコアが設定されています。
また,チームに分かれてはいますが個人ごとに競うことになります。


IOIのシラバス\cite{iois}は、IOIタスクに登場する可能性のあるトピックを掲載されており、
この本ではこのうちのほとんどのトピックを掲載しています。

IOIへの参加者は、国内コンテストを通じて選出されます。IOIの前には、
 Baltic Olympiad in Informatics (BOI)、
 the Central European Olympiad in Informatics (CEOI)、
 the Asia-Pacific Informatics Olympiad (APIO)などが開催されています。

 Croatian Open Competition in Informatics \cite{coci}
や、the USA Computing Olympiad \cite{usaco}.
など、将来のIOI参加者のためにオンラインコンテストを開催している国もあります。
さらに、ポーランドのコンテストで出された問題の大規模な
コレクションがオンラインで利用可能です\cite{main}。

\subsubsection{ICPC}

国際大学対抗プログラミングコンテスト(ICPC)は、
毎年開催される大学生を対象としたプログラミングコンテストです。
1チームは3名で構成され、
IOIとは異なり各チームに1台しかないコンピュータを使い競います。

ICPCはいくつかのステージで構成され、
勝ち進んだなチームがワールドファイナルに招待されます。
コンテストの参加者は数万人ですが、決勝の枠は限られています
\footnote{The exact number of final
slots varies from year to year; in 2017, there were 133 final slots.}。
決勝戦に進出するだけでも大きな意味を持ちます。

ICPCコンテストでは、
各チームが5時間の持ち時間で10問程度のアルゴリズムの問題を解き、
すべてのテストケースを効率的に解くことができた場合のみ、その解答が認められます。
コンテスト中は、他のチームの結果を見ることができます。
ただし、最後の1時間はスコアボードがフリーズしてしまい、 最後の投稿の結果を見ることができません。

ICPCで出題される可能性のあるテーマは、
IOIのようにあまり特定されていません。
ICPCではより多くの知識、特に数学的スキルが必要とされるとされています。

\subsubsection{Online contests}

誰でも参加できるオンラインコンテストは数多くあります。
最も活発なコンテストサイトはCodeforcesで、毎週のようにコンテストを開催しています。
Codeforcesでは、参加者を2つの部門に分け、初心者はDiv2、経験豊富なプログラマはDiv1で競います。
その他のコンテストサイトには、AtCoder、CS Academy、HackerRank、Topcoderがあります。

いくつかの企業には、オンラインでコンテストを開催し、
オンサイト(現地)で決勝を行うところもあります。
Facebook Hacker Cup、Google Code Jam、Yandex.Algorithmなどがその例です。
もちろん、企業はこうしたコンテストを採用活動にも活用しています。
コンテストで好成績を収めることは、自分のスキルを証明する良い方法でもあります。

\subsubsection{書籍 - Books}

競技プログラミングやアルゴリズムによる問題解決に焦点を当てた書籍は、(本書以外にも)すでにいくつかあります。

\begin{itemize}
\item S. S. Skiena and M. A. Revilla:
\emph{Programming Challenges: The Programming Contest Training Manual} \cite{ski03}
\item S. Halim and F. Halim:
\emph{Competitive Programming 3: The New Lower Bound of Programming Contests} \cite{hal13}
\item K. Diks et al.: \emph{Looking for a Challenge? The Ultimate Problem Set from
the University of Warsaw Programming Competitions} \cite{dik12}
\end{itemize}

最初の2冊は初心者向けで、最後の1冊は上級者向けの内容です。

もちろん、一般的なアルゴリズムの本も、競技志向のプログラマーには良いです。
人気のある本には、以下のようなものがあります。

\begin{itemize}
\item T. H. Cormen, C. E. Leiserson, R. L. Rivest and C. Stein:
\emph{Introduction to Algorithms} \cite{cor09}
\item J. Kleinberg and É. Tardos:
\emph{Algorithm Design} \cite{kle05}
\item S. S. Skiena:
\emph{The Algorithm Design Manual} \cite{ski08}
\end{itemize}
